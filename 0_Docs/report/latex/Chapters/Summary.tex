% Chapter 5

\chapter{Summary} % Main chapter title

\label{Chapter6} 

\section{Conclusion}

Predicting the propensity of a user to listen to music at a certain time, based on their recent listening history can be applied to a range of other areas such as the propensity to purchase, electricity usage, or the demands on a public transport system. In all these a good modeling of the temporal patterns helps with prediction.

Within the literature modeling the problem as a temporal point process is one of the most common methods and this research did likewise in several of the machine learning algorithms that were evaluted.

As music listening is typically long periods of non-play events, followed by sequential periods of play events, the problem becomes one of a balance between precision and recall. Improving recall requires predicting the start and end of a sequence while precision favours restricting predictions to when $t-1 = 1$. This effect is seen in our baseline model which scored 79\% on precision and 13\% on recall after 5-fold cross validation.

It is also seen in the linear SVM models which is less impacted by cases that fall close to the decision boundary as the start of and end of a sequence are likely to be.

From a practical perspective, predicting the start and end of series of events is what is of most interest. In this respect it seems non-linear models may perform better than linear ones with the RBF model achieving a 76\% recall score (while maintaining a high 77\% precision score), and the the RNN-LSTM model achiving 69\% and 70\% respectively.

\section{Future research}

Future research into the matter temporal point proceses should investigate whether non-linear models are indeed better at detecting both the start of an event process as well as noiser data in which the patterns may not be as obvious.

If non-linear models perform better at these problems, then, given the performance of RNN-LSTM models on a small set of data, they may be prove to perform exceptionally well in learning to model such temporal patterns with larger amounts of data and training. Currently the training time on an average laptop proved to be a bottleneck but with processing speeds constantly improving we may be entering an age where such models become practical in the commercial data science field. 

Finally the research touched upon (see Appndices) but not fully investigated the time it takes for models to adapt to new users. Hybrid models or advanced Bayesian models, that balance the observations coming from the user with wider prior knowlege, may be the key to learning the behaviour of individuals. After all the best predictor for a users behaviour, is observing the user themselves.