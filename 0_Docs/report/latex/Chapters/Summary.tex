% Chapter 5

\chapter{Summary} % Main chapter title

\label{Chapter6} 

\section{Conclusion}

Predicting the propensity of a user to listen to music at a certain time, based on their recent listening history can be applied to a range of other areas such as the propensity to purchase products, electricity usage, or the demands on a public transport system. The ability to accurately model these patterns is therefore of great significance to industry. Traditional temporal point processes modeling requires making assumptions about the data, such as the feature engineering done in this research, in order to build a prediction model and struggle with capturing highly non-linear patterns in a scalable manner. 

We applied a range of techniques to the task of modeling temporal point processes and used precision and recall as our evaluation measure. Our Baseline model which simply assumes $t = t-1$ performed the best, with a precision of 76\% and a recall of 80\%. As music listening is typically long periods of non-play events, followed by consequtive periods of play events, it is easy to achieve high scores using this heuristic. No models were able to exceed the baseline model. This points to the need to change the the problem definition. Rather than trying predict all Play events, it may be better to try and predict the start of a play sequence instead. Alternatively we may wish to predict the probability $n$ steps into the future.

Secondly, achieving a good balance between precision and recall is a challenge, and the RBF SVM model was the only one of our initial model that achieved this.

Finally we experimented with an RNN-LSTM model. As our literature review showed, the application of deep learning to temporal point processes is still in its infancy but shows promising signs. Our own research is inconclusive though positive enough to warrant further research. As mentioned, careful thought needs to be given to the problem definition in order to perform a reasonable assessment. 

Our take-away for the modeling of temporal processes in general, is to consider the following:
\begin{itemize}
	\item How much does the data differ at the macro vs. micro level?
	\item Are events singular events or occur in clumps?
	\item How balanced is the dataset?
	\item How well do simple heuristics explain the temporal patterns?
\end{itemize}

\section{Future research}

There are seveal directions one could follow to take this research forward.
\begin{enumerate}
	\item Attempt to predict the start of a play sequence
	\item Attempt to predict events $n$ steps into the future, where $t-1$ to $t-5$ are not available
	\item Investigate the variance between the temporal patterns of individual users and that of the population and models that can better deal with these
	\item Evaluation of more sophisticated RNN structures such as that described in our literature review \parencite{xiao2017modeling}.
	\item Evaluation of advanced Bayesian models such as Bayesian Logistic regression
	\item Application of Gaussian Point processes as way of getting around RBF scalability limitations
	\item Investigation into how quickly models can learn to model the behaviour of new users
\end{enumerate}