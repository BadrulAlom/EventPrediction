% Chapter 1

\chapter{Chapter Title Here} % Introduction

\label{Chapter1} % For referencing the chapter elsewhere, use \ref{Chapter1} 

%----------------------------------------------------------------------------------------

% Define some commands to keep the formatting separated from the content 
\newcommand{\keyword}[1]{\textbf{#1}}
\newcommand{\tabhead}[1]{\textbf{#1}}
\newcommand{\code}[1]{\texttt{#1}}
\newcommand{\file}[1]{\texttt{\bfseries#1}}
\newcommand{\option}[1]{\texttt{\itshape#1}}

%----------------------------------------------------------------------------------------

\section{Temporal patterns}
The impetus for this research was to estimate the liklihood of a person being interested in listening to music in the current time-period based on their only on their listening history. The implicit assumption is that a persons playlist history contains a temporal pattern such as a combination of daily and weekly schedule, that can be modelled. This concept can also be applied to other areas where a temporal pattern is thought to exist at an individual user level, such as the repeat purchase of household products, or the sleeping & eating habits of a person.

The problem is referred to as sequential event prediction in literature when modelled as a sequence of events and non-events, for every period.

\section{Structure of the report}
tbc
