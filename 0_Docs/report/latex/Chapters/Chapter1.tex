% Chapter 1

\chapter{Introduction} % Introduction

\label{Chapter1} % For referencing the chapter elsewhere, use \ref{Chapter1} 

%----------------------------------------------------------------------------------------

% Define some commands to keep the formatting separated from the content 
\newcommand{\keyword}[1]{\textbf{#1}}
\newcommand{\tabhead}[1]{\textbf{#1}}
\newcommand{\code}[1]{\texttt{#1}}
\newcommand{\file}[1]{\texttt{\bfseries#1}}
\newcommand{\option}[1]{\texttt{\itshape#1}}

%----------------------------------------------------------------------------------------

\section{Temporal patterns}
The impetus for this research was to estimate the liklihood of a person being interested in listening to music in the current time-period based on their only on their listening history. The implicit assumption is that a persons playlist history contains a temporal pattern such as a combination of daily and weekly schedule, that can be modelled. This concept can also be applied to other areas where a temporal pattern is thought to exist at an individual user level, such as the repeat purchase of household products, or the sleeping and eating habits of a person.

The problem is referred to as sequential event prediction in literature when modelled as a sequence of events and non-events, for every period. 

\section{The dataset}
The dataset being used in this analysis is the LastFM1k dataset containing the listening history of a thousand LastFM listeners.

The dataset contains the timestamp, userId, and trackId of users listening habits over a number of years (2005-2009).

\section{Objectives}
The objective of the research is to evaluate the effectiveness of several different techniques for determining the probability of a user listening to music in a given period. One such application of this would be home audio devices which could anticipate when a user would like to listen to music and play without user intervention.

As this is research is in the context of Data Science, it will focus on contrasting simple methods that can be easily explained and implemented, with more advanced methods.

\section{Structure of the report}
tbc
