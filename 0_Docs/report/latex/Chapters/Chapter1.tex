% Chapter 1

\chapter{Introduction} % Introduction

\label{Chapter1} % For referencing the chapter elsewhere, use \ref{Chapter1} 

%----------------------------------------------------------------------------------------

% Define some commands to keep the formatting separated from the content 
\newcommand{\keyword}[1]{\textbf{#1}}
\newcommand{\tabhead}[1]{\textbf{#1}}
\newcommand{\code}[1]{\texttt{#1}}
\newcommand{\file}[1]{\texttt{\bfseries#1}}
\newcommand{\option}[1]{\texttt{\itshape#1}}

%----------------------------------------------------------------------------------------

The modeling of event sequences is useful across industries. For instance the periods in which a customer makes an online purchase can help determine the optimal periods for target marketing. The times at which public transport users tend to travel can help better manage resources to meet demand. The times at which a medical illness re-occurs can help predict future episodes.
 
In all these cases modeling the temporal behaviour of the system is important in predicting the next event. At an aggregate level, behaviour may appear to be deterministic, such as the times at which peak rush-hour occurs, but such behaviour is often composed of thousands or millions of individual stochastic processes such as the decisions made by individuals as to whether to leave work at 5pm or continue working a little longer.

While the modeling of aggregate patterns is well understood, these models ofen breakdown when applied to customizing results for individual users. At this level the temporal patterns of an individual combined with the behaviour of the population may be a better preditor of event timing. For instance, sticking with the example above, the times at which a person has lunch during the day may help predict that they will finish work a little later.

while the area of product recommendation has received extensive attention in recent years, the area of recommendation timing less so. This research looks at how we can model the temporal behaviour of individuals, and whether deep learning is better able to learn these patterns than other methods.

\section{Motivation}

We take as our context for this research, the goal of estimating the probability that a user of a home-audio device would like to listen to music right now, based on their listening-event history. 

The raw data is a series of timestamps denoting when songs were played. The implicit assumption is that a persons playlist history contains a temporal pattern such as a combination of daily and weekly schedule, that can be modelled. This concept can also be applied to other areas where a temporal pattern is thought to exist at an individual user level, such as the repeat purchase of household products, or the sleeping and eating habits of a person.

The objective of the research is to evaluate the effectiveness of several different techniques for determining the probability of a user listening to music in a given period. One such application of this would be home audio devices which could anticipate when a user would like to listen to music and play without user intervention.

The research was guided by Emotech Ltd. the creators of Olly \parencite{Olly}.

\section{Point Processes}

One way of modeling the problem is as series of events and non-events known as a temporal point process. This has a rich history of methods as outlined in the literature review. 


\section{The dataset}
The dataset being used in this analysis is the LastFM1k dataset containing the listening history of a thousand LastFM listeners.

The dataset contains the timestamp, userId, and trackId of users listening habits over a number of years (2005-2009).

\section{Structure of the report}
tbc
