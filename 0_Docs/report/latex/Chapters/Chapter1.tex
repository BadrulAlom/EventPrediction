% Chapter 1

\chapter{Introduction} % Introduction

\label{Chapter1} % For referencing the chapter elsewhere, use \ref{Chapter1} 

%----------------------------------------------------------------------------------------

% Define some commands to keep the formatting separated from the content 
\newcommand{\keyword}[1]{\textbf{#1}}
\newcommand{\tabhead}[1]{\textbf{#1}}
\newcommand{\code}[1]{\texttt{#1}}
\newcommand{\file}[1]{\texttt{\bfseries#1}}
\newcommand{\option}[1]{\texttt{\itshape#1}}

%----------------------------------------------------------------------------------------

Event sequences are common in areas such as retail, finance, and utilities.
For example, the times at which a customer makes a purchase from an online retailer, the time between financial transactions in the stock market, and the times at which utility service customers make high use of gas and electricity. These are a particular category of event sequences in which the temporal factor could be said to be a key dimension in predicting the next occurence. In such cases, understanding and predicting user behaviors, based on purely the temporal aspect of their occurence, are of great practical, economic, and societal interest.

\section{Motivation}
The impetus for this research was to estimate the liklihood of a person being interested in listening to music in the current time-period based only on their listening history. The raw data is a series of timestamps denoting when songs were played. The implicit assumption is that a persons playlist history contains a temporal pattern such as a combination of daily and weekly schedule, that can be modelled. This concept can also be applied to other areas where a temporal pattern is thought to exist at an individual user level, such as the repeat purchase of household products, or the sleeping and eating habits of a person.

The objective of the research is to evaluate the effectiveness of several different techniques for determining the probability of a user listening to music in a given period. One such application of this would be home audio devices which could anticipate when a user would like to listen to music and play without user intervention.

The research was guided by Emotech Ltd. the creators of Olly \parencite{Olly}.

\section{Point Processes}

One way of modeling the problem is as series of events and non-events known as a temporal point process. This has a rich history of methods as outlined in the literature review. 


\section{The dataset}
The dataset being used in this analysis is the LastFM1k dataset containing the listening history of a thousand LastFM listeners.

The dataset contains the timestamp, userId, and trackId of users listening habits over a number of years (2005-2009).

\section{Structure of the report}
tbc
